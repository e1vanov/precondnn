\documentclass[a4paper, 12pt]{article}
% === ENCODINGS === 

\usepackage[english, russian]{babel}
\usepackage[T2A]{fontenc}
\usepackage[utf8]{inputenc}

% === MATH === 

% Some math fonts
\usepackage{amsfonts}

% Some math symbols
\usepackage{amssymb}

% Some math "making beautiful" stuff
\usepackage{mathtools}

% Some math fonts
\usepackage{mathrsfs}  

\numberwithin{equation}{section}

% === REFERENCES ===

\usepackage[sorting=none]{biblatex}
\addbibresource{sources.bib}


% === MY COMMANDS ===

\newcommand{\deriv}[2]{\frac{\partial #1}{\partial #2}}
\newcommand{\R}{\mathbb R}
\newcommand{\Row}{\sum\limits_{n=1}^\infty}
\newcommand{\Rowk}{\sum\limits_{k=1}^\infty}
\newcommand{\Prod}{\prod\limits_{n=1}^\infty}
\newcommand{\Prodk}{\prod\limits_{k=1}^\infty}
\newcommand{\eps}{\varepsilon}
\renewcommand{\phi}{\varphi}
\newcommand{\fall}{\:\forall\:}
\newcommand{\ex}{\:\exists\:}

% === MATH OPERATORS ===

\DeclareMathOperator{\const}{const}
\DeclareMathOperator{\Ker}{ker}
\DeclareMathOperator{\Image}{im}
\DeclareMathOperator{\Def}{def}
\DeclareMathOperator{\Rank}{rank}
\DeclareMathOperator{\Dim}{dim}
\DeclareMathOperator{\Argmin}{Argmin}
\DeclareMathOperator{\Tr}{tr}
\DeclareMathOperator{\Interior}{int}
\DeclareMathOperator{\Dom}{dom}
\DeclareMathOperator{\Aff}{aff}
\DeclareMathOperator{\Relint}{relint}

% === OTHER ===

% Indent in the begging of first par
\usepackage{indentfirst}



\begin{document}

\title{Нейросетевой подход к поиску\\ циркулянтных предобуславливателей \\для систем с теплицевыми матрицами}
\date{Осень 2024}

\maketitle

\tableofcontents

\section{Постановка задачи}

Имеется СЛАУ с теплицевой (пока симметричной матрицей):
\[
    \textbf T\textbf x = \textbf f,
\]
Требуется найти \textit{удачный} правый циркулянтный 
предобуславливатель $\textbf C^{-1}$.
Под удачным понимается предобуславливатель, ускоряющий сходимость
итерационных методов, например, GMRES.

\section{Подбор функционала}

Для начала мы хотим определить функционал / функционалы, которые будем использоваться для обучения сети.
Архитектуру пока отодвигаем в сторону, будем учить многослойный персептрон, который
гарантированно <<выучит все>>. 

Существует несколько общих мотиваций, из которых можно строить функционалы:

\begin{itemize}
    \item На скорость сходимости в итерационных методах
        влияет спектр матрицы $\textbf I - \textbf T\textbf C^{-1}$
        \begin{itemize}
            \item Зажимаем его в ноль равномерно,
                оптимизуем нормы: спектральный радиус,
                вторая норма ($\infty$-норма вектора
                сингулярных чисел), норма Фробениуса (2-норма вектора 
                сингулярных чисел).
            \item На самом деле мы не против небольшого количества
                выборосов среди собственных значений, то есть
                нас интересует (устраивает) малость спектра матрицы
                $\textbf I - \textbf T\textbf C^{-1} - \textbf R$, где
                $\textbf R$ -- матрица малого ранга. 
                В этом случае подходит ядерная норма (1-норма вектора
                сингулярных чисел), но она дорогая для
                вычисления. Звучит, как будто ее можно использовать
                как регуляризатор, но не на постоянной основе, а каждый
                батч / матрицу с вероятностью $p$ штрафовать за нее.
        \end{itemize}
    \item Можно учиться на быструю разрешимость конкретным
        методом, например, GMRES'ом. В этом подходе видятся 
        несколько проблем:
        \begin{itemize}
            \item Что считать функцией потерь?
                Невязку на $n$-ом шаге? Число итераций до сходимости?
                Штрафовать за число итераций или за распределение
                числа итераций?
            \item Кажется, что учить <<с нуля>> такую сеть
                будет сложно, поскольку понятие <<хороший спектр>>
                для сходимости GMRES немного размыто, мы скорее
                можем указать конкретные случаи, когда
                мы знаем, что сходимость должна быть быстрой, но
                есть ощущение (возможно, с подтверждением из линала), 
                что в начале обучения сеть будет
                много <<путаться>> и в итоге медленно учиться,
                возможно, можно выделить несколько <<голов>> сети
                и надеяться, что каждая выучит свою зависимость.
        \end{itemize}
\end{itemize}

\section{Преобразование Фурье?}

Мы знаем, что существует связь между
спектром теплицевой матрицы $\textbf T$ и частичной суммой ряда Фурье.
Отсюда кажется, что вход нейронной сети есть 
гармоники. Отсюда
возникает интересная связь:
если учить персептрон на гармониках, это
будет <<соответствовать>> обучению сверточной сети во временной области, так
как
\[
    \mathcal F^{-1}(\hat{\textbf{W}}\hat{\textbf{x}})
    =
    \mathcal F^{-1}(\hat{\textbf{W}}) * \mathcal F^{-1}(\hat{\textbf{x}})
    =
    \textbf W * \textbf x
    =
    \int \textbf W(t-\tau)\textbf x(\tau) d\tau
\]
То есть на уровне махания руками учить:
\[
    \textbf{T}\to \mathrm{Perceptron}(\boldsymbol\Theta) \to \textbf{C}^{-1}
\]
Равносильно тому, что учить
\[
    \textbf{T}\to\mathcal F^{-1}\to\mathrm{CNN}(\boldsymbol \Theta) 
    \to\mathcal F\to \textbf{C}^{-1}
\]
Равносильно с точки зрения выразительной способности,
но 
\begin{itemize}
    \item сверточные сети можно учить и применять на данных
        разных размерностей
    \item сверточные сети могут дать тот же скор при меньшем числе параметров
\end{itemize}

\end{document}
